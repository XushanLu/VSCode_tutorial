\documentclass{article}
\usepackage{amsmath}

\title{Geophysical Electromagnetic Methods}
\author{OpenAI}
\date{}

\begin{document}

\maketitle

\section{Introduction}
Geophysical Electromagnetic (EM) methods are a set of techniques used in geophysics to study the subsurface of the Earth. They are based on the principles of electromagnetism and the laws described by James Clerk Maxwell in the 19th century. In this paper, we will review the basic principles of EM methods and the equations that govern their behavior.

\section{Maxwell's Equations}
The basic equations of electromagnetism are known as Maxwell's equations. They describe the interaction between electric and magnetic fields and the behavior of electric charges. The equations are:

\begin{align}
\nabla \cdot \mathbf{E} &= \frac{\rho}{\epsilon_0} \
\nabla \cdot \mathbf{B} &= 0 \
\nabla \times \mathbf{E} &= -\frac{\partial \mathbf{B}}{\partial t} \
\nabla \times \mathbf{B} &= \mu_0 \mathbf{J} + \mu_0 \epsilon_0 \frac{\partial \mathbf{E}}{\partial t}
\end{align}

Where $\mathbf{E}$ is the electric field, $\mathbf{B}$ is the magnetic field, $\rho$ is the electric charge density, $\mathbf{J}$ is the current density, $\epsilon_0$ is the vacuum permittivity, and $\mu_0$ is the vacuum permeability.

\section{EM Methods}
EM methods can be divided into two categories: direct and indirect. Direct methods measure the primary electromagnetic field generated by an external source, while indirect methods measure the secondary fields induced by changes in the subsurface.

Some common direct EM methods include:
\begin{itemize}
\item Audio-magnetotellurics (AMT)
\item Controlled-Source Electromagnetic (CSEM)
\item Magnetotellurics (MT)
\end{itemize}

Indirect EM methods include:
\begin{itemize}
\item Electromagnetic Induction (EMI)
\item Ground Penetrating Radar (GPR)
\item Time Domain Electromagnetic (TDEM)
\end{itemize}

EM methods have a wide range of applications in geophysics, including the mapping of subsurface conductivity structures, the detection of hydrocarbons, and the imaging of geological structures.

\section{Conclusion}
In this paper, we have reviewed the basic principles of geophysical EM methods and the equations that govern their behavior. EM methods have become an important tool for geophysicists, providing valuable information about the subsurface of the Earth. Understanding the behavior of electromagnetic fields and the equations that describe them is essential for the proper use and interpretation of EM data.

\end{document}